\documentclass{article}

\usepackage{amssymb,amsmath,mleftright}

\begin{document}

General model assumptions:
\begin{itemize}
\item the fluid density and viscosity are constant in time and space (implies incompressibility);
\item the fluid flow is laminar;
\item the column is operated under constant conditions (e.g., temperature, flow rate);
\item the process is isothermal (i.e., there are no thermal effects);
\item diffusion does not depend on the concentration of the components, viscosity of the fluid, or pressure;
\item the fluid in the particles is stagnant (i.e., there is no convective flow);
\item the porous particles are spherical, rigid, and do not move;
\item the particles are homogeneous and of uniform porosity;
\item all components access the same particle volume;
\item the partial molar volumes are the same in mobile and solid phase;
\item the binding model parameters do not depend on pressure and are constant along the column;
\item the solvent is not adsorbed;
\end{itemize}


Specific model assumptions:
\begin{itemize}
\item the particles form a continuum inside the column (i.e., there is interstitial and particle volume at every point in the column);
\item the column is radially symmetric and homogeneous (i.e., concentration profiles and parameters only depend on the axial position);
\item the interstitial liquid phase concentration is spatially constant on the outer particle surface;
\item inside the particles, adsorbed molecules may diffuse along the surface of the stationary phase (surface diffusion);
\item the film around the particles does not limit mass transfer. That is, we assume $k^{\mathrm{f}}_i = \infty$)
\item adsorption and desorption happen on a much faster time scale than the other mass transfer processes (e.g., convection, diffusion). Hence, we consider them to be equilibrated instantly, that is, to always be in (local) equilibrium
\end{itemize}


\section*{General Rate Model}
Consider a cylindrical column of length $L > 0$ packed with spherical particles, and observed over a time interval $(0, T^{\mathrm{end}})$.
In the interstitial volume, mass transfer is governed by the following convection equations in $(0, T^\mathrm{end})\times (0, L)$ and for all components $i\in\{1, \dots, N^{\mathrm{c}} \}$
\begin{align}
\varepsilon^{\mathrm{c}} \frac{\partial c^{\mathrm{b}}_i}{\partial t} = - u \frac{\partial \left( \varepsilon^{\mathrm{c}} c^{\mathrm{b}}_i \right)}{\partial z}- \left(1 - \varepsilon^{\mathrm{c}} \right) \frac{3}{R^{\mathrm{p}}} \left( \varepsilon^{\mathrm{p}} D^{\mathrm{p}}_{i} \left. \frac{\partial c^{\mathrm{p}}_{i}}{\partial r} + (1 - \varepsilon^{\mathrm{p}}) D^{\mathrm{s}}_{i} \frac{\partial c^{\mathrm{s}}_{i}}{\partial r}\right)\right|_{r = R^{\mathrm{p}}},
\end{align}
with boundary conditions

\begin{alignat}{2}
u c_{\mathrm{in},i} &= \left.\left( u c^{\mathrm{b}}_i \right)\right|_{z=0} & &\qquad\text{on }(0, T^{\mathrm{end}}).
\end{alignat}
Here, $t\in (0, T^{\mathrm{end}})$ is the time coordinate, $z\in (0, L)$ is the axial cylinder coordinate, $R^\mathrm{p}> 0$ is the particle radius, $c^{\mathrm{b}}_i\colon (0, T^\mathrm{end})\times (0, L) \mapsto \mathbb{R}$ is the bulk liquid concentration, $\varepsilon^{\mathrm{c}}\in (0, 1)$ is the column porosity, and $u> 0$ is the interstitial velocity.
In the particles, mass transfer is governed by diffusion-reaction equations in $ (0, T^\mathrm{end}) \times (0, L)\times (0, R^{\mathrm{p}})$ and for all components

\begin{align}
\frac{\partial c^{\mathrm{p}}_{i}}{\partial t} + \frac{1 - \varepsilon^{\mathrm{p}}}{\varepsilon^{\mathrm{p}}} \frac{\partial c^{\mathrm{s}}_{i}}{\partial t}&= \frac{1}{r^2} \frac{\partial }{\partial r} \left( r^2 D_{i}^{\mathrm{p}} \frac{\partial c^{\mathrm{p}}_{i}}{\partial r} \right) - \frac{1 - \varepsilon^{\mathrm{p}}}{\varepsilon^{\mathrm{p}}} \frac{1}{r^2} \frac{\partial }{\partial r} \left( r^2 D_{i}^{\mathrm{s}} \frac{\partial c^{\mathrm{s}}_{i}}{\partial r} \right) , \\
0 &= f^{\mathrm{bind}}_{i}\left( \vec{c}^{\mathrm{p}}, \vec{c}^{\mathrm{s}} \right) ,
\end{align}

with boundary conditions
\begin{align}
- \left. \left( \varepsilon^{\mathrm{p}} D^{\mathrm{p}}_{i} \frac{\partial c^{\mathrm{p}}_{i}}{\partial r} + (1 - \varepsilon^{\mathrm{p}}) D^{\mathrm{s}}_{i} \frac{\partial c^{\mathrm{s}}_{i}}{\partial r} \right) \right|_{r=0}
&= 0, \\
\left. c^{\mathrm{p}}_{i} \right|_{r = R^{\mathrm{p}}_{}} &= c^{\mathrm{b}}_i.\end{align}
Here, $r$ is the radial particle coordinate, $c^{\mathrm{p}}_{i}\colon  (0, T^\mathrm{end}) \times (0, L)\times (0, R^{\mathrm{p}}) \mapsto \mathbb{R}$ is the particle liquid concentration, $c^{\mathrm{s}}_{i}\colon  (0, T^\mathrm{end}) \times (0, L)\times (0, R^{\mathrm{p}}) \mapsto \mathbb{R}$ is the particle solid concentration, $\varepsilon^{\mathrm{p}}\in (0, 1)$ is the particle porosity, $D^\mathrm{p}_{i}> 0$ is the particle diffusion coefficient, $D^\mathrm{s}_{i}\geq 0$ is the surface diffusion coefficient, $f^\mathrm{bind}_{i}$ is the adsorption isotherm function, $\vec{c}^\mathrm{p}$ is the particle liquid components vector, and $\vec{c}^\mathrm{s}$ is the particle solid components vector.
Consistent initial values for all solution variables (concentrations) are defined at $t = 0$.
\end{document}